\phantomsection\label{f7652a3e}
\nointerlineskip\nointerlineskip\begin{minted}[autogobble,samepage]{python}
""""Solution to Chapter 3 problem DN"""
import numpy as np
import matplotlib.pyplot as plt
\end{minted}

\phantomsection\label{fd92f6de}

\phantomsection\label{111288b4}
\subsubsection{Introduction}\label{introduction}

This program defines several mathematical functions as vectorized
functions that can handle NumPy array inputs and plots them over the
given domain using Matplotlib.

\phantomsection\label{ac335079}
\subsubsection{Define Mathematical
Functions}\label{define-mathematical-functions}

Define $f(x) = x^2 + 3 x + 9$:

\phantomsection\label{12bfbd65}
\nointerlineskip\nointerlineskip\begin{minted}[autogobble,samepage]{python}
def f(x: np.ndarray) -> np.ndarray:
    return np.tanh(4 * np.sin(x))
\end{minted}

\phantomsection\label{85a19e11}
Define $g(x) = 1 + \sin^2 x$:

\phantomsection\label{91f761bb}
\nointerlineskip\nointerlineskip\begin{minted}[autogobble,samepage]{python}
def g(x: np.ndarray) -> np.ndarray:
    return np.sin(np.sqrt(x))
\end{minted}

\phantomsection\label{39fb9a87}
Define $h(x, y) = e^{-3 x} + \ln y$:

\phantomsection\label{70b88c82}
\nointerlineskip\nointerlineskip\begin{minted}[autogobble,samepage]{python}
def h(x: np.ndarray) -> np.ndarray:
    return np.where(x >= 0, np.exp(-x) * np.sin(2 * np.pi * x), 0)
\end{minted}

\phantomsection\label{ab1a2f18}
\subsubsection{Plotting}\label{plotting}

Define a plotting function:

\phantomsection\label{d49ccae9}
\nointerlineskip\nointerlineskip\begin{minted}[autogobble,samepage]{python}
def plotter(fig, fun, limits, labels):
    x = np.linspace(limits[0], limits[1], 201)
    fig.gca().plot(x, fun(x))
    fig.gca().set_xlabel(labels[0])
    fig.gca().set_ylabel(labels[1])
    return fig
\end{minted}

\phantomsection\label{1f9cc092}
Plot $f(x)$:

\phantomsection\label{ae14fca3}
\nointerlineskip\nointerlineskip\begin{minted}[autogobble,samepage]{python}
fig, ax = plt.subplots()
plotter(fig, fun=f, limits=(-5, 8), labels=("$x$", "$f(x)$"))
\end{minted}

\phantomsection\label{76f34bb1}
\gdef\graphicslist{}%
\begin{figure}[htbp]
\centering
\includegraphics[]{examples/publishing_with_makefile/figure-0.pgf}
\caption{}
\label{fig:publishing_with_makefile-figure-0}
\end{figure}

\phantomsection\label{6e605b5e}
Plot $g(x)$:

\phantomsection\label{3049ff64}
\nointerlineskip\nointerlineskip\begin{minted}[autogobble,samepage]{python}
fig, ax = plt.subplots()
plotter(fig, fun=g, limits=(0, 100), labels=("$x$", "$g(x)$"))
\end{minted}

\phantomsection\label{fd4bfb5d}
\gdef\graphicslist{}%
\begin{figure}[htbp]
\centering
\includegraphics[]{examples/publishing_with_makefile/figure-1.pgf}
\caption{}
\label{fig:publishing_with_makefile-figure-1}
\end{figure}

\phantomsection\label{4a022c7a}
Plot $h(x)$:

\phantomsection\label{feea135a}
\nointerlineskip\nointerlineskip\begin{minted}[autogobble,samepage]{python}
fig, ax = plt.subplots()
plotter(fig, fun=h, limits=(-2, 6), labels=("$x$", "$h(x)$"))
\end{minted}

\phantomsection\label{e940f698}
\gdef\graphicslist{}%
\begin{figure}[htbp]
\centering
\includegraphics[]{examples/publishing_with_makefile/figure-2.pgf}
\caption{}
\label{fig:publishing_with_makefile-figure-2}
\end{figure}

\phantomsection\label{e2ab5739}
\nointerlineskip\nointerlineskip\begin{minted}[autogobble,samepage]{python}
plt.show()
\end{minted}
